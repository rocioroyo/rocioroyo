\documentclass[runningheads]{llncs}

                %%%PREAMBULO%%%

\usepackage[T1]{fontenc}
\usepackage[utf8]{inputenc}
\usepackage[spanish]{babel}
\usepackage{graphicx}
\parindent =1cm
\authorrunning{Power Rangers ;)}
\begin{document}

\title{Descarbonización}
%
\author{Abel Feriozzi \and Rocio Royo \and Sofia Landi \and Jose Solanes \and Ignacio Gambino \and Renata Gori \and Nabil Affar\and Guido Scopel (corrector)}

\institute {Universidad Nacional de Cuyo}
%
\maketitle 

\begin{abstract}

En esta sección hablaremos de como la sociedad puede avanzar hacia un producción industrial descarbonizada.

\keywords {Descarbonizacion \and Industria \and Global}

\end{abstract}

%%%%%%%%%%%%%%%%%%%%%%%%%%%%%%%%%%%%%%%%%%%%%%%%%
\begin{flushleft}
\section*{Estrategias energéticas sostenibles y limpias}
\end{flushleft}

\subsection*{Nexo entre alimentos, energía y agua}
La agricultura aporta aproximadamente el 25 por ciento de todas las emisiones de gases del efecto invernadero (GEI), y aproximadamente el 80 por ciento de estos provienen de productos de la agricultura animal como carne, leche, y huevos. La industria agrícola tiene importantes impactos en el medio ambiente, ya que la producción ganadera representa el 70 por ciento de todas las tierras agrícolas y el 30 por ciento de la superficie terrestre. La infraestructura hídrica utiliza cantidades considerables de energía en adquisiciones, transporte y purificación, y es una fuente importante de emisiones industriales de GEI.\\ \par 
Los sistemas de alimentos, energía y agua están profundamente interconectados. Necesitamos energía y agua para cultivar alimentos. La energía requiere de agua para extraer y funcionar. Los sistemas de agua requieren energía para obtener, mover, limpiar y eliminar. Estas interconexiones ilustran cómo nuestros sistemas de agua, alimentos y energía son interdependientes.\\ \par
\textbf{Definición:} \textit{El nexo entre alimentos, energía y agua es una frase que se usa para describir los procesos biofísicos, naturales, sociales y de comportamiento interconectados entre sí.}\\ \par
¿Cómo podemos diseñar estos sistemas interconectados que funcionen para resolver problemas en el nexo alimento-energía-agua (A-E-A)? Los sistemas AEA comprenden procesos físicos (como construcción de infraestructura y nuevas tecnologías para una utilización más eficiente de los recursos), procesos naturales (como los ciclos biogeoquímicos e hidrológicos), procesos biológicos (como agroecosistemas, estructura y productividad) y procesos sociales y de comportamiento (como la toma de decisiones y la gobernanza). Los sistemas AEA no tienen una delimitación y pueden definirse apropiadamente en una amplia gama de escalas temporales y espaciales- de local a regional a global.
\newpage
\textit{La agricultura y la silvicultura producen alrededor del 25 por ciento de los gases de efecto invernadero anuales a nivel mundial. Otro 25 por ciento de los gases de efecto invernadero provienen de la electricidad y el calor.}\\ \par
La utilización de energía solar puede traer grandes beneficios, reduciendo los  problemas que trae la producción alimenticia en el nexo AEA. Al aumentar el despliegue de la cantidad de energía solar, tanto la intensidad de GEI, como la cantidad de agua utilizada para la producción alimenticia, se pueden reducir. Incluso mínimas reducciones en la cantidad de agua utilizada puede tener importantes consecuencias en la producción de cultivos.\\ \par
Otra área importante en la que puede haber oportunidades en el nexo del sistema AEA es la producción mundial de carne, que aumentó a 350 millones de toneladas en 2014 de 78 millones de toneladas en 1961. La producción ganadera es la principal fuente de emisión de carbono, con contribuciones particularmente altas en emisiones de metano. Éste se encuentra naturalmente en el centro de la ganadería, lo que se conoce como fermentación, y en el abono del ganado, donde combinados emiten el 37 por ciento del metano anual. Más de la cuarta parte de todo el metano que ingresa a la atmósfera proviene de la cría de ganado, originada por el hombre. El metano es un GEI más potente que el dióxido de carbono, con 23 a 25 veces la capacidad de capturar el calor del dióxido de carbono. El ganado también contribuye con el 64 por ciento de las emisiones de amoníaco, principal causante de la lluvia ácida y acidificación de la tierra y las vías fluviales. A  su vez, este también consume mucha agua, requiriendo más de 900 litros para producir un galón de leche.

\subsection*{Conceptos de sustentabilidad y justicia para la energía solar futura}
Al surgir diferentes problemas en todo el oeste de Estados Unidos referidos a distintos proyectos solares a escala de servicios públicos, se idearon distintas soluciones que no involucren tierras sin perturbar, es decir, aprovechando infraestructura ya existente y protegiendo los espacios abiertos.\\ \par
Esta “preservación de la tierra” se basa en oportunidades de ubicación y se utiliza para describir áreas ya utilizadas tales como tierras contaminadas con sal, cuerpos de agua y particularmente aguas residuales, las cuales son mucho más extensas de lo que se creía. La iniciativa RE-Powering America's Land de la EPA de Estados Unidos reveló a través de sus estudios,muchas tierras disponibles que  podrían satisfacer la demanda máxima de energía. También se descubrió que muchas tierras utilizadas para la agricultura y antiguas zonas industriales ya han sido abandonadas o bien resultan inapropiadas para la utilización en otros fines.\\ \par
Los proyectos planteados de campos solares podrían llevarse a cabo como una remodelación o como un esfuerzo de revitalización. Existe el riesgo de que se presenten problemas con respecto a la contaminación de los suelos gracias a productos químicos utilizados en ellas y esto puede mantener a los inversores alejados de tales proyectos. El impacto generado por las granjas solares puede llegar a ser muy importante, pero de lo que realmente depende es del uso de las tierras existentes y de otros factores como la calidad del hábitat.\\ \par
Dentro de lo que serían áreas agrícolas, lo que podrían mejorar estas granjas solares son los servicios de polinización ya que podría ser una fuente de ingreso adicional o simplemente ayudar la polinización de tierras adyacentes. Los polinizadores, tanto las abejas agrícolas como las nativas, se encuentran en extinción actualmente. Para mejorar dicha situación se puede agregar forraje y así mantenerlos saludables beneficiando también la productividad agrícola del paisaje\\ \par
Este texto trata sobre los principales desafíos que implica el uso de la tierra con respecto a las energías renovables. Propone diferentes formas de utilizar la energía solar pero a su vez evitar el cambio de uso de la tierra para lograr ofrecer beneficios ambientales o servicios ecosistémicos. Otra solución propuesta es utilizar energía fotovoltaica en los pabellones de estacionamientos, manteniendo los automóviles frescos y reduciendo el efecto de isla que provoca el calor urbano, esto se logra ya que los pabellones fotovoltaicos no tienen la misma masa térmica que el pavimento.

\subsection*{Desarrollo de estrategias de descarbonización}
Hay muchas formas de ver el futuro de patrones de uso de la energía y los escenarios de impacto ambiental asociados al mismo. El Back-Casting es un enfoque que construye un escenario para transformar un sistema energético para que alcance los requisitos de un objetivo. Los economistas utilizan una restricción de impacto y buscan el resultado más eficiente económicamente.

\subsection*{Conceptos críticos para las estrategias de energía sostenible}
La descarbonización de nuestra economía y nuestros sistemas energéticos requerirá cooperación a nivel internacional para lograr la mayor velocidad de cambio posible ya que actualmente la mayoría de los países industrializados tienen altos niveles de emisiones de GEI. Para ello se debería incurrir en nuevas políticas y tecnologías a nivel mundial, y por sobre todo un gran compromiso social de las grandes empresas. Esta “carrera” de la humanidad hacia la descarbonización requerirá que la tecnología avance rápidamente.\\ \par
La inclusión de nuevas tecnologías en las empresas deberá cumplir (en el mejor de los casos) con dos condiciones; la primera y principal es que logre una mayor sustentabilidad de los procesos industriales a través de nueva maquinaria, gestión de los productos al terminar su vida útil y la responsabilidad extendida del productor. La segunda condición es lograr una mayor optimización en los procesos con el fin de obtener mayores aumentos de valor en las empresas.\\ \par
Por último, un factor clave a tener en cuenta es la llamada “transición justa”, que nos indica que algunos países serán más susceptibles al cambio o afectarán en mayor medida su economía para mal, por lo que debe lograrse un cambio que produzca la menor cantidad de efectos negativos en el mundo.

\subsection*{Sinergias tecno-ecológicas}
En un planeta donde conviven muchas personas y especies, es fundamental tomar decisiones acertadas sobre las transiciones energéticas de la tierra para maximizar la productividad y minimizar el impacto de la infraestructura energética. La expansión energética fomentó la idea de sinergias tecno-ecológicas para destacar las oportunidades de crear infraestructuras generativas. Esta transición hacia la descarbonización de las economías es una tarea a nivel mundial que comprenderá promulgar leyes para fomentar el desarrollo de las energías renovables. Las industrias como el cemento y la agricultura deberán reducir el uso de carbono en el futuro.\\ \par
A diferencia de otros sistemas energéticos, la energía solar generada a partir de la fotovoltaica puede combinarse con el entorno construido, las infraestructuras humanas o los paisajes de trabajo, como la agricultura, los pastizales y la acuicultura. Esto supone una oportunidad para maximizar los múltiples beneficios medioambientales que se pasan por alto del despliegue de la energía solar a través de las sinergias tecno-ecológicas, un marco para diseñar relaciones mutuamente beneficiosas entre los sistemas tecnológicos y ecológicos\\ \par
Un grupo internacional de investigadores liderado por la Dra. Rebecca R. Hernández, profesora de la Universidad de California, y otras 11 organizaciones (incluida la autora de este libro) creen que las sinergias tecno-ecológicas representan un marco importante para la producción sostenible de electricidad. Un artículo en Nature Sustainability proporciona información de las posibilidades de potenciar las ventajas ecológicas y tecnológicas de la energía solar. Se identificaron dieciséis tipos diferentes de instalaciones solares con posibles sinergias tecno-ecológicas, incluidas las instalaciones sobre terrenos previamente perturbados y sobre el agua. Este enfoque daría lugar a la conservación de las zonas naturales no perturbadas, conservando plantas y animales, incluidos los polinizadores, biodiversidad y ciclos intactos de retención de carbono.\\ \par
El marco de sinergias tecno-ecológicas puede servir tanto para acelerar la transición como para fomentar una planificación inteligente de las instalaciones solares (Hernandez et al. 2019). Al proporcionar más pruebas de los beneficios de la energía solar, el marco puede aplicarse en los debates normativos y políticos para contrarrestar los ataques de las compañías de combustibles fósiles y las compañías eléctricas tradicionales que se resisten a mantener el statu quo. El marco también ofrece la oportunidad de maximizar cuidadosamente los beneficios de tantas instalaciones solares como sea posible. Al considerar las crisis climáticas, además de los desafíos que plantea el suministro de electricidad a una población mundial en rápido crecimiento con crecientes demandas per cápita, se pueden evitar las consecuencias no deseadas de una transición energética masiva.\\ \par
Las corporaciones y los gobiernos están comenzando a hacer cambios al comprometerse con la energía 100 por ciento limpia y desinvertir en combustibles fósiles. Estas políticas y prácticas garantizan que la energía solar seguirá creciendo rápidamente en algunos lugares. La implementación de planes para adoptar energía limpia y obligar a los lentos a actuar plantea grandes interrogantes como la rapidez con que se producirá la transición, y dónde y cómo se desarrollará la energía solar. La forma en que una organización transita al uso de energía 100 por ciento limpia, incluyendo peticiones de beneficios adicionales para los ecosistemas y la agricultura, se está volviendo cada vez más importante.
Entre estas sinergias tecno-ecológicas, el equipo caracterizó 20 beneficios específicos, que van desde la resiliencia de la red hasta el ahorro de tierra y el hábitat de los polinizadores, y creó una matriz para comunicar estos resultados sinérgicos. Un ejemplo importante que los autores identificaron es la oportunidad de la energía solar en tierras degradadas, como minas abandonadas y terrenos baldíos. Los autores descubrieron que las tierras degradadas en Estados Unidos representan casi el doble de la superficie de California. De estos terrenos, los más degradados, como los de la EPA Superfund, podrían producir más de 1,6 millones de GWh al año de electricidad solar fotovoltaica. Eso es más del 35 por ciento del consumo total de electricidad de Estados Unidos en 2015.\\ \par
La agricultura es otra oportunidad de sinergia tecno-ecológica. El equipo calcula que se podrían producir 21.000 km2 o 1.350 GWh en Estados Unidos utilizando estas tierras. Esto también podría estar directamente vinculado a los sistemas de riego. Basándose en una investigación anterior del laboratorio de la Dra. Rebecca Hernández (Hoffacker et al. 2017), el equipo estima 39 TWh al año de potencial energético en 104 km2 de embalses agrícolas en el Valle Central de California. Esto podría aportar el 15 por ciento de la electricidad anual de California y ahorrar 0,12 km3 de agua al año.\\ \par
La restauración de las tierras degradadas puede mejorar los servicios de los ecosistemas y ofrece un potencial de beneficios para el secuestro de carbono. Una investigación del Servicio Geológico de Estados Unidos señala que los paisajes áridos serán una fuente de emisiones hasta el año 2.100 debido al cambio de uso del suelo. Unas mejores prácticas de uso del suelo podrían convertir esta fuente en un sumidero mediante una mejor conservación.\\ \par
Los sistemas alimentarios ofrecen una increíble oportunidad para las sinergias tecno-ecológicas solares. Los "sistemas agrivoltaicos" o los paneles colocados en la misma zona que la producción agrícola tienen beneficios como el aumento de la búsqueda de recursos por parte de los polinizadores autóctonos, el aumento de la eficiencia en el uso del agua y la prevención de la erosión del suelo. La infraestructura de energía solar también puede alterar las condiciones microclimáticas que benefician a la producción general de los cultivos con una mayor eficiencia en el uso del agua y mantienen los sistemas fotovoltaicos más frescos y con un funcionamiento más eficiente. La ubicación conjunta de la agricultura y los sistemas fotovoltaicos también da lugar a un mayor rendimiento de los alimentos y la energía solar combinados en comparación con la producción de alimentos y energía solar por separado.\\ \par
\textbf{Definición:} \textit{La energía fotovoltaica flotante es una energía fotovoltaica integrada con materiales que le permiten flotar e instalarse en lagos, embalses, estanques de tratamiento de aguas residuales, bahías oceánicas y otros cursos de agua marinos.}\\ \par
En los sistemas acuáticos, los autores examinan las múltiples ventajas de la "fotovoltaica" para los paneles fijados a balsas que flotan en el agua. Estos sistemas pueden proporcionar 11 resultados beneficiosos potenciales, que van desde la reducción del crecimiento de las algas hasta la prevención de la pérdida de agua por evaporación, un beneficio que es particularmente importante para considerar la colocación de la energía solar sobre los acuíferos, los embalses y los sistemas de tratamiento o desalinización del agua. El equipo señaló que, según investigaciones anteriores, 4.500 m2 de paneles fotovoltaicos producen 425.000 kWh y ahorran 5.000 m3 de agua al año gracias a la evaporación evitada.\\ \par
En las construcciones, destacamos las sinergias solares en los tejados y otros espacios urbanizados en o cerca de donde vive y trabaja la gente. Los paneles en los tejados pueden aislar los edificios para mejorar el ahorro de energía y favorecer la salud y el confort de las personas. Los paneles pueden enfriar los edificios disminuyendo la necesidad de aire acondicionado en los calurosos meses de verano.\\ \par
Las tecnologías solares emergentes, como los paneles solares transparentes no hacen más que poner en manifiesto cómo los avances en la ciencia de los materiales abren más puertas a las sinergias tecno-ecológicas. Las continuas innovaciones en el ámbito de la energía solar y eólica también proporcionarán aún más oportunidades .El equipo caracterizó estos beneficios en sus documentos como un prometedor "trampolín” para la integración de las sinergias tecno-ecológicas de la energía solar en la industria y en la sociedad. Estas sinergias requieren sus propias políticas, incentivos y subvenciones, además de las ya existentes para otras tecnologías de energía limpia como la energía solar.\\ \par
En EEUU hay más de 800.000 km2 de terrenos degradados para el desarrollo de la energía solar: terrenos baldíos, vertederos, minas abandonadas y tierras agrícolas contaminadas y abandonadas. Se espera realizar un análisis costo-beneficio de la fijación de tarifas eléctricas y otros procesos de fijación de valores que afecten a los mercados de energía solar distribuidos y, en última instancia, ofrecer beneficios a aquellas empresas que se instalen en estos terrenos degradados.

\subsection*{Avanzando en la transición energética hacia la descarbonización}
El libro busca que los lectores obtengan su propia visión, mediante debates y herramientas. Las transiciones energéticas tendrán gran valor en el resto del siglo, ya sea hablando de sustentabilidad, cambio climático o acceso a la energía. Se necesitan conocimientos de ciencia, tecnología, artes y humanidades. Uno de los conceptos que hay que comprender es el de resiliencia, que representa lo bien que un sistema puede recuperarse de un cambio.\\ \par
Los proyectos de combustibles fósiles se harán algunas preguntas como:
\begin{itemize}

\item ¿Cómo mantendrán la actividad económica estos lugares?
\item ¿Dónde van a conseguir nuevos empleos? 
\item ¿Cómo van a reemplazar los ingresos y los impuestos que a veces pagan por importantes recursos comunitarios?

\end{itemize}

Hay que buscar la manera de que los trabajadores de estas empresas no pierdan sus empleos durante estos cambios.\\ \par
La transición energética se da por cambios sociales, culturales, política y desafíos técnicos. Los mismos se deberán afrontar con mucha capacitación y pensamientos interdisciplinarios. Ya hace más de una década que varios países se han comprometido a utilizar un 100 por ciento de energías renovables, limpias o bajas en carbono.

\end{document}